

% {{cookiecutter.first_name}} {{cookiecutter.last_name}} - {{cookiecutter.title}}

\documentclass[11pt]{article}

%
% Packages
%

% Page setup
\usepackage[top=1in, bottom=1in, left=1in, right=1in]{geometry}
\usepackage{fancyhdr}

% Characters
\usepackage[T1]{fontenc}
\usepackage[utf8]{inputenc}

% Math
\usepackage{amsmath}
\usepackage{amssymb}

% Algorithm pseudocode
\usepackage{algorithm}
\usepackage[noend]{algpseudocode}

% Images and captions
\usepackage{graphicx}
\usepackage{caption}
\usepackage{float}

% Lists
\usepackage[shortlabels]{enumitem}

% Hyperlinks
\usepackage{hyperref}
\hypersetup {
  % colorlinks=true,
  % urlcolor=blue,
}

% Code font
\usepackage{courier}

% Text
\usepackage{setspace}
\usepackage[export]{adjustbox}
\usepackage{textcomp}

%
% Math commands
%

\DeclareMathOperator{\E}{\mathbb{E}}
\DeclareMathOperator{\R}{\mathbb{R}}

%
% Code listing
%

\usepackage{listings}
\usepackage{xcolor}
\definecolor{code-green}{rgb}{0,0.6,0}
\definecolor{code-purple}{HTML}{8959a8}
\definecolor{code-light-grey}{HTML}{f2f2f2}
\definecolor{code-dark-grey}{HTML}{606060}
\lstset{
    backgroundcolor=\color{code-light-grey},
    basicstyle=\footnotesize\ttfamily,
    commentstyle=\color{code-dark-grey}\ttfamily\it,
    keywordstyle=\color{code-purple}\ttfamily\bf,
    stringstyle=\color{code-green},
    numberstyle=\tiny\color{code-dark-grey},
    numbersep=6pt,
    numbers=left,
    keepspaces=true,
    showspaces=false,
    showstringspaces=false,
    showtabs=false,
    tabsize=2,
    breakatwhitespace=false,
    breaklines=true,
    captionpos=b,
    rulecolor=\color{code-light-grey},
    linewidth=1.0\textwidth
}

% ------------------------------------------------------------------------------

\setlength\parindent{0pt}
\setlength\parskip{5pt}
\pagestyle{fancy}
\fancyhf{}
\lhead{ {{cookiecutter.first_name}} {{cookiecutter.last_name}} }
\rhead{ {{cookiecutter.title}} - Page \thepage}

\begin{document}

\section{Equations}

\begin{align*}
  x &= y \\
  y &= x \\
\end{align*}

\newpage

\section{Code}

% \lstinputlisting[language=Python, firstline=1, lastline=6]{foo.py}

\newpage

\section{Pseudocode}

\begin{algorithm}
\begin{algorithmic}[1]
  \Procedure{foo}{$bar, baz$}
    \State \textbf{\textit{Input:}} Stuff
    \State \textbf{\textit{Output:}} More Stuff
    \While{$stuff$ is $stuff$}
      \State $bar \gets$ $baz$
      \State $bar = baz$
    \EndWhile
  \EndProcedure
\end{algorithmic}
\end{algorithm}

\newpage

\section{Image}

% \begin{figure}[H]
%   \includegraphics[width=\linewidth]{boat.jpg}
%   \caption{Caption}
%   \label{fig:label1}
% \end{figure}

\end{document}



% {{cookiecutter.first_name}} {{cookiecutter.last_name}} - {{cookiecutter.title}}

\documentclass[aspectratio=169]{beamer}

%
% Fonts
%

% Roboto light for main font; available in texlive-fonts-extra on Ubuntu.
\usepackage[light]{roboto}
% Math font is serif (i.e. looks like article math).
\usefonttheme[onlymath]{serif}
% Monospace font for code listings.
\usepackage{inconsolata}

%
% Packages
%

% Characters
\usepackage[T1]{fontenc}
\usepackage[utf8]{inputenc}

% Math
\usepackage{amsmath}
\usepackage{amssymb}

% Algorithm pseudocode
\usepackage{algorithm}
\usepackage[noend]{algpseudocode}

% Images and captions
\usepackage{graphicx}
\usepackage{caption}
\usepackage{float}

% Hyperlinks
\usepackage{hyperref}
\hypersetup {
  % colorlinks=true,
  % urlcolor=blue,
}

%
% Math commands
%

\DeclareMathOperator{\E}{\mathbb{E}}
\DeclareMathOperator{\R}{\mathbb{R}}

%
% Code listing
%

\usepackage{listings}
\usepackage{xcolor}
\definecolor{code-green}{rgb}{0,0.6,0}
\definecolor{code-purple}{HTML}{8959a8}
\definecolor{code-light-grey}{HTML}{f2f2f2}
\definecolor{code-dark-grey}{HTML}{606060}
\lstset{
    backgroundcolor=\color{code-light-grey},
    basicstyle=\footnotesize\ttfamily,
    commentstyle=\color{code-dark-grey}\ttfamily\it,
    keywordstyle=\color{code-purple}\ttfamily\bf,
    stringstyle=\color{code-green},
    numberstyle=\tiny\color{code-dark-grey},
    numbersep=6pt,
    numbers=left,
    keepspaces=true,
    showspaces=false,
    showstringspaces=false,
    showtabs=false,
    tabsize=2,
    breakatwhitespace=false,
    breaklines=true,
    captionpos=b,
    rulecolor=\color{code-light-grey},
    linewidth=1.0\textwidth,
    xleftmargin=8pt, % Adds some spacing on the sides.
    xrightmargin=8pt,
}

%
% Beamer settings
%

% Based on Madrid theme

% Color theme
\usecolortheme{whale}
\usecolortheme{orchid}

% Outer theme
\useoutertheme{infolines}
\setbeamertemplate{headline}[default]

% Inner theme.
\useinnertheme{rectangles}

% Bullets
\setbeamertemplate{itemize items}[default]
\setbeamertemplate{enumerate items}[default]

% Custom title page.
\setbeamerfont{title}{size=\huge}
\setbeamercolor{title}{bg=white,fg=black}
\setbeamerfont{subtitle}{shape=\itshape}
\defbeamertemplate*{title page}{clean}[1][]
{
  \begin{columns}
  \column{0.6\textwidth}
  \vbox{}
  \vfill
  \begingroup
    \centering
    \vskip2em
    \begin{beamercolorbox}[sep=5pt,left,#1]{title}
      \usebeamerfont{title}\inserttitle\par%
      \ifx\insertsubtitle\@empty%
      \else%
      \vskip0.5em
        {\usebeamerfont{subtitle}\usebeamercolor[fg]{subtitle}\insertsubtitle\par}%
      \fi%
    \end{beamercolorbox}%
    \begin{beamercolorbox}[sep=5pt,left,#1]{author}
      \usebeamerfont{author}\insertauthor
    \end{beamercolorbox}
    \begin{beamercolorbox}[sep=5pt,left,#1]{date}
      \usebeamerfont{date}\insertdate
    \end{beamercolorbox}\vskip0.5em
    % TODO
    % \begin{beamercolorbox}[sep=8pt,center,#1]{institute}
    %   \usebeamerfont{institute}\insertinstitute
    % \end{beamercolorbox}
    {\usebeamercolor[fg]{titlegraphic}\inserttitlegraphic\par}
  \endgroup
  \vfill
  \column{0.4\textwidth}
  (Some picture or other filler text)
  \end{columns}
}

% ------------------------------------------------------------------------------

\title[About Beamer] %optional
{About the Beamer class in presentation making}

\subtitle{A short story}

\author[Arthur, Doe] % (optional, for multiple authors)
{A.~B.~Arthur\inst{1} \and J.~Doe\inst{2}}

\institute[VFU] % (optional)
{
  \inst{1}%
  Faculty of Physics\\
  Very Famous University
  \and
  \inst{2}%
  Faculty of Chemistry\\
  Very Famous University
}

\date[VLC 2013] % (optional)
{Very Large Conference, April 2013}

% Adds TOC at beginning of each section.
% \AtBeginSection[]
% {
%   \begin{frame}
%     \frametitle{Table of Contents}
%     \tableofcontents[currentsection]
%   \end{frame}
% }


\begin{document}


\frame{\titlepage}


\begin{frame}
  \frametitle{Table of Contents}
  \tableofcontents
\end{frame}


\section{This is a section}


\begin{frame}[fragile]
\frametitle{Code}
\begin{lstlisting}[frame=single, language=Python]
def foobar(n):
  """Do stuff..."""
  x = 5000
  for i in range(n):
    # Very complicated.
    print("Hello World!")
\end{lstlisting}
\end{frame}


\begin{frame}
\frametitle{Equations}
A random equation:
$$ Q(s,a) = r(s,a) + \gamma\E_{a'|s \sim \pi}\left[Q(s',a')\right] $$
\end{frame}


\begin{frame}
\frametitle{Sample frame with bullets}
\begin{itemize}
    \pause
  \item This is a bullet.
    \pause
  \item This is another bullet.
    \pause
  \item And yet another.
\end{itemize}
\end{frame}


\section{This is another section}


\begin{frame}[fragile]
\frametitle{Two Columns}
\begin{columns}

\column{0.5\textwidth}
This is some text in the first column.

\column{0.5\textwidth}
This is some code in the second column.
\begin{lstlisting}[frame=single, language=Python]
def foobar(n):
  """Do stuff..."""
  x = 5000
  for i in range(n):
    # Very complicated.
    print("Hello World!")
\end{lstlisting}

\end{columns}
\end{frame}


\end{document}



% {{cookiecutter.first_name}} {{cookiecutter.last_name}} - {{cookiecutter.title}}

\iffalse

## TODO

- [ ] foo

## Outline

- Intro
- Body 1
- Body 2
- Body 3
- Conclusion

\fi


\documentclass[12pt]{article}

% Margin - 1 inch on all sides
\usepackage[letterpaper]{geometry}
\usepackage{times}
\geometry{top=1.0in, bottom=1.0in, left=1.0in, right=1.0in}

% Doublespacing
\usepackage{setspace}
\doublespacing

% Rotating tables (e.g. sideways when too long)
\usepackage{rotating}


% Fancy-header package to modify header/page numbering (insert last name)
\usepackage{fancyhdr}
\pagestyle{fancy}
\lhead{}
\chead{}
\rhead{ {{cookiecutter.last_name}} \thepage}
\lfoot{}
\cfoot{}
\rfoot{}
\renewcommand{\headrulewidth}{0pt}
\renewcommand{\footrulewidth}{0pt}
% To make sure we actually have header 0.5in away from top edge
% 12pt is one-sixth of an inch. Subtract this from 0.5in to get headsep value
\setlength\headsep{0.333in}

% Works cited environment
% (to start, use \begin{workscited...}, each entry preceded by \bibent)
\newcommand{\bibent}{\noindent \hangindent 0.5in}
\newenvironment{workscited}{\newpage \begin{center} Works Cited\end{center}}{\newpage }


% Begin document
\begin{document}
\begin{flushleft}

% First page name, class, etc
{{cookiecutter.first_name}} {{cookiecutter.last_name}}\\
{{cookiecutter.instructor}}\\
{{cookiecutter.class}}\\
{{cookiecutter.date}}\\

% Title
\begin{center}
{{cookiecutter.title}}
\end{center}

% Changes paragraph indentation to 0.5in
\setlength{\parindent}{0.5in}
% Begin body of paper here


Hello World


% Works cited
\begin{workscited}

% \bibent \textit{The Oxford Study Bible}. Edited by M. Jack Suggs, Katharine Doob
%   Sakenfeld, and James R. Mueller, Oxford University Press, 1992.

\end{workscited}

\end{flushleft}
\end{document}
\}



% {{cookiecutter.first_name}} {{cookiecutter.last_name}} - {{cookiecutter.title}}

\documentclass[11pt]{article}

\usepackage[top=0.5in, bottom=0.5in, left=0.5in, right=0.5in]{geometry}
\usepackage{titlesec}

%
% Font
%

\usepackage{fontspec}

% the main font, with all features on
\setmainfont
  [ ExternalLocation ,
    Mapping          = tex-text ,
    Numbers          = OldStyle ,
    Ligatures        = {Common,Contextual} ,
    BoldFont         = texgyrepagella-bold.otf ,
    ItalicFont       = texgyrepagella-italic.otf ,
    BoldItalicFont   = texgyrepagella-bolditalic.otf ]
  {texgyrepagella-regular.otf}

% Hyperlinks
\usepackage{hyperref}
\hypersetup {
  colorlinks=true,
  urlcolor=blue,
}

% No page break after title
\let\endtitlepage\relax

% Format section titles
\titleformat{\section}[block]{\bf\centering}{\thesection}{1em}{\MakeUppercase}
\titlespacing{\section}{0pt}{*0.5}{*0.5}

% Hyphenation
\hyphenpenalty=10000

\begin{document}

\frenchspacing
\pagenumbering{gobble}

\begin{center}
  \textbf{\uppercase{\Large{ {{cookiecutter.title}} }\\
  \normalsize{ {{cookiecutter.first_name}} {{cookiecutter.last_name}} }}}\\
  \vspace{-3mm}
\end{center}

Hello World

\iffalse

## TODO

- [ ] foo

## Outline

- Intro
- Body 1
- Body 2
- Body 3
- Conclusion

\fi

\end{document}


